\documentclass[conference,compsoc]{IEEEtran}
\usepackage[T1]{fontenc}
\usepackage[utf8]{inputenc}
\usepackage{tabularx,ragged2e,booktabs,caption}
\newcolumntype{C}[1]{>{\Centering}m{#1}}
\renewcommand\tabularxcolumn[1]{C{#1}}
% *** CITATION PACKAGES ***
%
\ifCLASSOPTIONcompsoc
  % IEEE Computer Society needs nocompress option
  % requires cite.sty v4.0 or later (November 2003)
  \usepackage[nocompress]{cite}
\else
  % normal IEEE
  \usepackage{cite}
\fi
% cite.sty was written by Donald Arseneau
% V1.6 and later of IEEEtran pre-defines the format of the cite.sty package
% \cite{} output to follow that of the IEEE. Loading the cite package will
% result in citation numbers being automatically sorted and properly
% "compressed/ranged". e.g., [1], [9], [2], [7], [5], [6] without using
% cite.sty will become [1], [2], [5]--[7], [9] using cite.sty. cite.sty's
% \cite will automatically add leading space, if needed. Use cite.sty's
% noadjust option (cite.sty V3.8 and later) if you want to turn this off
% such as if a citation ever needs to be enclosed in parenthesis.
% cite.sty is already installed on most LaTeX systems. Be sure and use
% version 5.0 (2009-03-20) and later if using hyperref.sty.
% The latest version can be obtained at:
% http://www.ctan.org/pkg/cite
% The documentation is contained in the cite.sty file itself.
%
% Note that some packages require special options to format as the Computer
% Society requires. In particular, Computer Society  papers do not use
% compressed citation ranges as is done in typical IEEE papers
% (e.g., [1]-[4]). Instead, they list every citation separately in order
% (e.g., [1], [2], [3], [4]). To get the latter we need to load the cite
% package with the nocompress option which is supported by cite.sty v4.0
% and later.





% *** GRAPHICS RELATED PACKAGES ***
%
\ifCLASSINFOpdf
  % \usepackage[pdftex]{graphicx}
  % declare the path(s) where your graphic files are
  % \graphicspath{{../pdf/}{../jpeg/}}
  % and their extensions so you won't have to specify these with
  % every instance of \includegraphics
  % \DeclareGraphicsExtensions{.pdf,.jpeg,.png}
\else
  % or other class option (dvipsone, dvipdf, if not using dvips). graphicx
  % will default to the driver specified in the system graphics.cfg if no
  % driver is specified.
  % \usepackage[dvips]{graphicx}
  % declare the path(s) where your graphic files are
  % \graphicspath{{../eps/}}
  % and their extensions so you won't have to specify these with
  % every instance of \includegraphics
  % \DeclareGraphicsExtensions{.eps}
\fi
% graphicx was written by David Carlisle and Sebastian Rahtz. It is
% required if you want graphics, photos, etc. graphicx.sty is already
% installed on most LaTeX systems. The latest version and documentation
% can be obtained at: 
% http://www.ctan.org/pkg/graphicx
% Another good source of documentation is "Using Imported Graphics in
% LaTeX2e" by Keith Reckdahl which can be found at:
% http://www.ctan.org/pkg/epslatex
%
% latex, and pdflatex in dvi mode, support graphics in encapsulated
% postscript (.eps) format. pdflatex in pdf mode supports graphics
% in .pdf, .jpeg, .png and .mps (metapost) formats. Users should ensure
% that all non-photo figures use a vector format (.eps, .pdf, .mps) and
% not a bitmapped formats (.jpeg, .png). The IEEE frowns on bitmapped formats
% which can result in "jaggedy"/blurry rendering of lines and letters as
% well as large increases in file sizes.
%
% You can find documentation about the pdfTeX application at:
% http://www.tug.org/applications/pdftex





% *** MATH PACKAGES ***
%
\usepackage{amsmath}
% A popular package from the American Mathematical Society that provides
% many useful and powerful commands for dealing with mathematics.
%
% Note that the amsmath package sets \interdisplaylinepenalty to 10000
% thus preventing page breaks from occurring within multiline equations. Use:
%\interdisplaylinepenalty=2500
% after loading amsmath to restore such page breaks as IEEEtran.cls normally
% does. amsmath.sty is already installed on most LaTeX systems. The latest
% version and documentation can be obtained at:
% http://www.ctan.org/pkg/amsmath





% *** SPECIALIZED LIST PACKAGES ***
%
\usepackage{algorithm}
\usepackage{algpseudocode}
\usepackage{float}

\renewcommand{\algorithmicrequire}{\textbf{Input:}}
\renewcommand{\algorithmicensure}{\textbf{Output:}}
% algorithmic.sty was written by Peter Williams and Rogerio Brito.
% This package provides an algorithmic environment fo describing algorithms.
% You can use the algorithmic environment in-text or within a figure
% environment to provide for a floating algorithm. Do NOT use the algorithm
% floating environment provided by algorithm.sty (by the same authors) or
% algorithm2e.sty (by Christophe Fiorio) as the IEEE does not use dedicated
% algorithm float types and packages that provide these will not provide
% correct IEEE style captions. The latest version and documentation of
% algorithmic.sty can be obtained at:
% http://www.ctan.org/pkg/algorithms
% Also of interest may be the (relatively newer and more customizable)
% algorithmicx.sty package by Szasz Janos:
% http://www.ctan.org/pkg/algorithmicx


\usepackage[T1]{fontenc} 
\usepackage[utf8]{inputenc}




% *** ALIGNMENT PACKAGES ***
%
\usepackage{array}
% Frank Mittelbach's and David Carlisle's .sty patches and improves
% the standard LaTeX2e array and tabular environments to provide better
% appearance and additional user controls. As the default LaTeX2e table
% generation code is lacking to the point of almost being broken with
% respect to the quality of the end results, all users are strongly
% advised to use an enhanced (at the very least that provided by array.sty)
% set of table tools. array.sty is already installed on most systems. The
% latest version and documentation can be obtained at:
% http://www.ctan.org/pkg/array


% IEEEtran contains the IEEEeqnarray family of commands that can be used to
% generate multiline equations as well as matrices, tables, etc., of high
% quality.




% *** SUBFIGURE PACKAGES ***
%\ifCLASSOPTIONcompsoc
%  \usepackage[caption=false,font=footnotesize,labelfont=sf,textfont=sf]{subfig}
%\else
%  \usepackage[caption=false,font=footnotesize]{subfig}
%\fi
% subfig.sty, written by Steven Douglas Cochran, is the modern replacement
% for subfigure.sty, the latter of which is no longer maintained and is
% incompatible with some LaTeX packages including fixltx2e. However,
% subfig.sty requires and automatically loads Axel Sommerfeldt's caption.sty
% which will override IEEEtran.cls' handling of captions and this will result
% in non-IEEE style figure/table captions. To prevent this problem, be sure
% and invoke subfig.sty's "caption=false" package option (available since
% subfig.sty version 1.3, 2005/06/28) as this is will preserve IEEEtran.cls
% handling of captions.
% Note that the Computer Society format requires a sans serif font rather
% than the serif font used in traditional IEEE formatting and thus the need
% to invoke different subfig.sty package options depending on whether
% compsoc mode has been enabled.
%
% The latest version and documentation of subfig.sty can be obtained at:
% http://www.ctan.org/pkg/subfig




% *** FLOAT PACKAGES ***
%
%\usepackage{fixltx2e}
% fixltx2e, the successor to the earlier fix2col.sty, was written by
% Frank Mittelbach and David Carlisle. This package corrects a few problems
% in the LaTeX2e kernel, the most notable of which is that in current
% LaTeX2e releases, the ordering of single and double column floats is not
% guaranteed to be preserved. Thus, an unpatched LaTeX2e can allow a
% single column figure to be placed prior to an earlier double column
% figure.
% Be aware that LaTeX2e kernels dated 2015 and later have fixltx2e.sty's
% corrections already built into the system in which case a warning will
% be issued if an attempt is made to load fixltx2e.sty as it is no longer
% needed.
% The latest version and documentation can be found at:
% http://www.ctan.org/pkg/fixltx2e


%\usepackage{stfloats}
% stfloats.sty was written by Sigitas Tolusis. This package gives LaTeX2e
% the ability to do double column floats at the bottom of the page as well
% as the top. (e.g., "\begin{figure*}[!b]" is not normally possible in
% LaTeX2e). It also provides a command:
%\fnbelowfloat
% to enable the placement of footnotes below bottom floats (the standard
% LaTeX2e kernel puts them above bottom floats). This is an invasive package
% which rewrites many portions of the LaTeX2e float routines. It may not work
% with other packages that modify the LaTeX2e float routines. The latest
% version and documentation can be obtained at:
% http://www.ctan.org/pkg/stfloats
% Do not use the stfloats baselinefloat ability as the IEEE does not allow
% \baselineskip to stretch. Authors submitting work to the IEEE should note
% that the IEEE rarely uses double column equations and that authors should try
% to avoid such use. Do not be tempted to use the cuted.sty or midfloat.sty
% packages (also by Sigitas Tolusis) as the IEEE does not format its papers in
% such ways.
% Do not attempt to use stfloats with fixltx2e as they are incompatible.
% Instead, use Morten Hogholm'a dblfloatfix which combines the features
% of both fixltx2e and stfloats:
%
% \usepackage{dblfloatfix}
% The latest version can be found at:
% http://www.ctan.org/pkg/dblfloatfix




% *** PDF, URL AND HYPERLINK PACKAGES ***
%
\usepackage{url}
% url.sty was written by Donald Arseneau. It provides better support for
% handling and breaking URLs. url.sty is already installed on most LaTeX
% systems. The latest version and documentation can be obtained at:
% http://www.ctan.org/pkg/url
% Basically, \url{my_url_here}.




% *** Do not adjust lengths that control margins, column widths, etc. ***
% *** Do not use packages that alter fonts (such as pslatex).         ***
% There should be no need to do such things with IEEEtran.cls V1.6 and later.
% (Unless specifically asked to do so by the journal or conference you plan
% to submit to, of course. )


% correct bad hyphenation here
\hyphenation{op-tical net-works semi-conduc-tor}


\begin{document}
%
% paper title
% Titles are generally capitalized except for words such as a, an, and, as,
% at, but, by, for, in, nor, of, on, or, the, to and up, which are usually
% not capitalized unless they are the first or last word of the title.
% Linebreaks \\ can be used within to get better formatting as desired.
% Do not put math or special symbols in the title.
\title{Influence Maximization Problem}


% author names and affiliations
% use a multiple column layout for up to three different
% affiliations
\author{\IEEEauthorblockN{YingZhou  11610701}
\IEEEauthorblockA{Computer Science and Engineering\\
SUSTech\\
11610701@mail.sustc.edu.cn}}


% conference papers do not typically use \thanks and this command
% is locked out in conference mode. If really needed, such as for
% the acknowledgment of grants, issue a \IEEEoverridecommandlockouts
% after \documentclass

% for over three affiliations, or if they all won't fit within the width
% of the page (and note that there is less available width in this regard for
% compsoc conferences compared to traditional conferences), use this
% alternative format:
% 
%\author{\IEEEauthorblockN{Michael Shell\IEEEauthorrefmark{1},
%Homer Simpson\IEEEauthorrefmark{2},
%James Kirk\IEEEauthorrefmark{3}, 
%Montgomery Scott\IEEEauthorrefmark{3} and
%Eldon Tyrell\IEEEauthorrefmark{4}}
%\IEEEauthorblockA{\IEEEauthorrefmark{1}School of Electrical and Computer Engineering\\
%Georgia Institute of Technology,
%Atlanta, Georgia 30332--0250\\ Email: see http://www.michaelshell.org/contact.html}
%\IEEEauthorblockA{\IEEEauthorrefmark{2}Twentieth Century Fox, Springfield, USA\\
%Email: homer@thesimpsons.com}
%\IEEEauthorblockA{\IEEEauthorrefmark{3}Starfleet Academy, San Francisco, California 96678-2391\\
%Telephone: (800) 555--1212, Fax: (888) 555--1212}
%\IEEEauthorblockA{\IEEEauthorrefmark{4}Tyrell Inc., 123 Replicant Street, Los Angeles, California 90210--4321}}




% use for special paper notices
%\IEEEspecialpapernotice{(Invited Paper)}




% make the title area
\maketitle

% As a general rule, do not put math, special symbols or citations
% in the abstract
%\begin{abstract}

%\end{abstract}

% no keywords




% For peer review papers, you can put extra information on the cover
% page as needed:
% \ifCLASSOPTIONpeerreview
% \begin{center} \bfseries EDICS Category: 3-BBND \end{center}
% \fi
%
% For peerreview papers, this IEEEtran command inserts a page break and
% creates the second title. It will be ignored for other modes.
\IEEEpeerreviewmaketitle



\section{Preliminaries}
\subsection{Problem Description}
The \textit{influence maximization problem} (IMP) normally asks for $k$ finding a
small subset of $k$ nodes (referred to as seed set) in a social network $G$ that
could maximize the spread of influence which is the expected number of nodes that are
influenced by the nodes in the seed set in a cascade manner (ISE). The project includes two parts which are influence speed estimate (ISE) and influence maximization problem (IMP).

\subsection{Problem Application}
The \textit{influence maxmization problem} (IMP) has been extensively studied recently due to its potential commercial value. A well-known application is the case that a company wants to spread the adoption of a new product from some initially selected adopters through the social links between users.

% An example of a floating figure using the graphicx package.
% Note that \label must occur AFTER (or within) \caption.
% For figures, \caption should occur after the \includegraphics.
% Note that IEEEtran v1.7 and later has special internal code that
% is designed to preserve the operation of \label within \caption
% even when the captionsoff option is in effect. However, because
% of issues like this, it may be the safest practice to put all your
% \label just after \caption rather than within \caption{}.
%
% Reminder: the "draftcls" or "draftclsnofoot", not "draft", class
% option should be used if it is desired that the figures are to be
% displayed while in draft mode.
%
%\begin{figure}[!t]
%\centering
%\includegraphics[width=2.5in]{myfigure}
% where an .eps filename suffix will be assumed under latex, 
% and a .pdf suffix will be assumed for pdflatex; or what has been declared
% via \DeclareGraphicsExtensions.
%\caption{Simulation results for the network.}
%\label{fig_sim}
%\end{figure}

% Note that the IEEE typically puts floats only at the top, even when this
% results in a large percentage of a column being occupied by floats.


% An example of a double column floating figure using two subfigures.
% (The subfig.sty package must be loaded for this to work.)
% The subfigure \label commands are set within each subfloat command,
% and the \label for the overall figure must come after \caption.
% \hfil is used as a separator to get equal spacing.
% Watch out that the combined width of all the subfigures on a 
% line do not exceed the text width or a line break will occur.
%
%\begin{figure*}[!t]
%\centering
%\subfloat[Case I]{\includegraphics[width=2.5in]{box}%
%\label{fig_first_case}}
%\hfil
%\subfloat[Case II]{\includegraphics[width=2.5in]{box}%
%\label{fig_second_case}}
%\caption{Simulation results for the network.}
%\label{fig_sim}
%\end{figure*}
%
% Note that often IEEE papers with subfigures do not employ subfigure
% captions (using the optional argument to \subfloat[]), but instead will
% reference/describe all of them (a), (b), etc., within the main caption.
% Be aware that for subfig.sty to generate the (a), (b), etc., subfigure
% labels, the optional argument to \subfloat must be present. If a
% subcaption is not desired, just leave its contents blank,
% e.g., \subfloat[].


% An example of a floating table. Note that, for IEEE style tables, the
% \caption command should come BEFORE the table and, given that table
% captions serve much like titles, are usually capitalized except for words
% such as a, an, and, as, at, but, by, for, in, nor, of, on, or, the, to
% and up, which are usually not capitalized unless they are the first or
% last word of the caption. Table text will default to \footnotesize as
% the IEEE normally uses this smaller font for tables.
% The \label must come after \caption as always.
%
%\begin{table}[!t]
%% increase table row spacing, adjust to taste
%\renewcommand{\arraystretch}{1.3}
% if using array.sty, it might be a good idea to tweak the value of
% \extrarowheight as needed to properly center the text within the cells
%\caption{An Example of a Table}
%\label{table_example}
%\centering
%% Some packages, such as MDW tools, offer better commands for making tables
%% than the plain LaTeX2e tabular which is used here.
%\begin{tabular}{|c||c|}
%\hline
%One & Two\\
%\hline
%Three & Four\\
%\hline
%\end{tabular}
%\end{table}


% Note that the IEEE does not put floats in the very first column
% - or typically anywhere on the first page for that matter. Also,
% in-text middle ("here") positioning is typically not used, but it
% is allowed and encouraged for Computer Society conferences (but
% not Computer Society journals). Most IEEE journals/conferences use
% top floats exclusively. 
% Note that, LaTeX2e, unlike IEEE journals/conferences, places
% footnotes above bottom floats. This can be corrected via the
% \fnbelowfloat command of the stfloats package.



\section{Methodology}

\subsection{Notation}

\begin{table}[H]
\caption{Notation}
\centering\begin{tabular}{|>{\centering\arraybackslash}p{15mm}|p{65mm}|} 
	\hline
	\textbf{Notation} & \textbf{Description}\\
	\hline
	$G = (V,E)$ &  a social network $G$ with a node set $V$ and an edge set $E$\\ 
	\hline
	n, m & the numbers of nodes and edges in $G$, respectively \\
	\hline
	k & the size of the seed set for influence maximization \\ 
	\hline
	$R$ & the set of RR sets generated by IMM's sampling phase\\
	\hline
	$\theta$ & the number of RR sets in $R$ \\
	\hline
	$F_R(s)$ & the faction of RR sets in $R$ that are covered by a node set S\\
	\hline
\end{tabular}
\end{table}


\subsection{Data Structure} 
The main data structures I used contain dictionary, queue, set and heap et. Dictionary is used to store the edges and the weight of each edge since using adjacency matrix is memory wasting if all graphs given are sparse. Queue is applied in multi-processes part for workers to get and put the jobs. Set replacing list records active vertices to save initializing time because set in Python is implemented in hash table. Finally, lazy update optimization with heap implemented in node selection part. 
			


\subsection{Model Design}
Both ISE and IMP are accelerated with 8 processes. ISE is strictly implemented according to the pseudo code of tutorial lecture. The novel point about IMM is that, in a social network, a influential person starred by many people may not follow same number of people. In that case, the sub-graph of converse graph is much smaller than that of original graph. It is simple and straightforward in implementation process as the description of the paper \cite{IMM} is clear and specific.    


\subsection{Detail of Algorithm}
\subsubsection{Influence Spread Estimate}
Given the seed set $S$, the diffusion processes unfolds in discrete rounds as follow: At round 0, all nodes in $S$ are active and the others are inactive. In the subsequent rounds, the newly activated nodes will try to activate their neighbors. The process will stop when no more nodes get activated. 
There are two fundamental diffusion models are specified in this project including \textit{Independent Cascade} (IC) and \textit{Linear Threshold} (LT) .\\
\noindent \textbf{Independent Cascade} When a node $u$ gets activated, initially or by another node, it has a single chance to activate each inactivate neighbor $v$ with the probability proportional to the edge weight $w(u, v)$.\\
\noindent \textbf{Linear Threshold} At the beginning, each node $v$ selects a random threshold $\theta_v$ uniformly at random in range [0,1]. Since the second round, an inactive node $v$ becomes activated if $\sum{activated\; neighbors\;u} W(u,v) \geq \theta_v$. The weight of the edge $(u,v)$ equals $\dfrac{1}{d_{in}(v)}$.


\begin{algorithm}[H]
	\caption{Independent Cascade}
	\begin{algorithmic}[1]
		\Function{IC}{}
		\State Initialoze ActivitySet as\{SeedSet\}
		\State count $\gets$ ActivitySet.length
		\While{ActivitySet is not empty}
		\State Set newAcitivitySet as $\emptyset$ 
		\For {each seed in AcitivitySet}
			\For {each inactive neighbor in seed}
			\State Seed tries to activate neighbors with probability
				\If{activated}
				\State Add neighbor into activate to newActicitySet
				\EndIf
			\EndFor
		\EndFor
		\State count $\gets$ count + newActivitySet.length
		\State ActivitySet $\gets$ newActivitySet
		\EndWhile
		\EndFunction
		\State return count
	\end{algorithmic}
\end{algorithm}

\begin{algorithm}[H]
	\caption{Linear Threshold}
	\begin{algorithmic}[1]
		\Function{LT}{}
		\State Initialize ActivitySet as \{SeedSet\}
		\State Randomly sample thresholds
		\State count $\gets$ ActivitySet.length
		\While{ActivitySet is not empty}
		\State Set newAcitivitySet as $\emptyset$ 
		\For {each seed in AcitivitySet}
			\For {each inactive neighbor in seed}
			\State Calculate the weights of activated neighbors as w\_total
			\If{w\_total $\geq$ neighbor.thresh}
			\State Update the state as Active
			\State newActivitySet
			\EndIf
			\State Add neighbor into activate to newActicitySet
			\EndFor
		\EndFor
		\EndWhile
		\EndFunction
	\end{algorithmic}
\end{algorithm}


\subsubsection{Influence Maximazation Problem}
All the details and proofs about the algorithm I used in IMP are referred  Y.Tang's work \cite{IMM}. I just implemented it and decided which data structures to use to make code run faster. 
\begin{algorithm}
	\caption{IMM}
	\begin{algorithmic}[1]
		\Function{IMM}{}
		\State $l = l \cdot (1 + \log(2)/\log(n)$
		\State $R = $ Sampling ($G,k,\epsilon,l$)
		\State $S_k = $ NodeSelection ($R,k$)
		\EndFunction	
	\end{algorithmic}
\end{algorithm}



\begin{algorithm}[H]
	\caption{Sampling}
	\begin{algorithmic}[1]
		\Function{Sampling}{$G,k,\epsilon,l$}
		\State Initialize a set $R = \emptyset$ and an integer $LB = 1$
		\State Let $\prime{\epsilon} = \sqrt{2} \cdot \epsilon$ 
		\For{$i = 1$ to $\log_2 n$}
		\State $x = n/2^i$
		\State $\theta_i = \lambda\prime / x$, where $\lambda\prime$ is as defined in equation 1
		\While{$\|{R}\| \leq \theta_i$}
		\State Select a node $v$ from $G$ uniformly at random
		\State Generate an RR set for $v$, and insert it into $R$
		\EndWhile
		\State Let $S_i = $ NodeSelection($R$)
		\If {$n \cdot F+R(S_i) \geq (1 + \epsilon\prime) \cdot x$}
		\State $LB = n \cdot F_R(S_i)/(1+\epsilon\prime)$ 
		\State \textbf{break}
		\EndIf	
		\EndFor
		\State Let $\theta = \lambda^\ast/LB$, where $\lambda^\ast$ is as defined as in Equation 2
		\EndFunction		
	\end{algorithmic}
\end{algorithm}

\begin{algorithm}
	\caption{NodeSelection}
	\begin{algorithmic}[1]
		\Function{NodeSelection}{$R,k$}
		\State Initialize a node set $S_k = \emptyset$
		\For{$i = 1$ to $k$}
		\State Identify the vertex $v$ that maximizes $F_R(S_k \cup v) - F_R$
		\State Insert $v$ into $S_k$
		\EndFor
		\EndFunction
	\end{algorithmic}
\end{algorithm}

\section{Empirical Verification}

\subsection{Dataset}
\begin{table}[H]
	\caption{Datasets}
	\centering\begin{tabular}{|c|c|c|c|} 
		\hline		
		\textbf{Name} & $n$ & $m$ & \textbf{Type} \\
		\hline
		\textit{NetHEPT} & 15.2K & 32.2K & undirected \\
		\hline
		\textit{Epinion} & 39.0M & 84.0K & directed \\
		\hline
		\textit{Twitter} & 81.3K & 1.8M & undirected  \\
		\hline
	\end{tabular}
\end{table}


\subsection{Hyperparameters}
\par 
\begin{equation}
\lambda\prime = \dfrac{(2 + \dfrac{2}{3}\epsilon\prime) \cdot (\log\binom{n}{k}) + l \cdot \log n + \log \log_2 n) \cdot n}{\epsilon\prime^2}
\end{equation}
\begin{equation}
\lambda^\ast = 2n \cdot ((1-1/e)\cdot\alpha+\beta)^2\cdot\epsilon^{-2}
\end{equation}
\begin{equation}
\alpha = \sqrt{l\log n + \log 2}
\end{equation}
\begin{equation}
\beta = \sqrt{(1 - 1/e) \cdot (\log\binom{n}{k} + l\log n + log 2)}
\end{equation}

Most parameters are proposed in Y.Tang;s work \cite{IMM}. The default setting of Algorithm 3 is $\epsilon = 0.1$ and $l = 1$. The maximization algorithm the paper presented provides a $(1-1/2-\epsilon)$-approximate solution with at least $1-1/n^l$ probability.  The $\epsilon$ and $l$ setting is a trade-off between solution quality and time cost, and is set as the same as most people.

\subsection{Performance Measure}
\par The performances of accuracy and efficiency are considered. I measure the accuracy of results by comparing time and results with best performance in OJ and the efficiency presented by running in larger instances.
The code is written in Python, compiled using Python 3.6.5 :: Anaconda. Only NumPy package is extra imported. Ubuntu 18.04.1,   
16 processors, each processor with 2 threads, 32G RAM,  Intel(R) Xeon(R) CPU E5-2620 v4 @ 2.10GHz.
\subsection{Experimental Result}

\begin{minipage}{\linewidth}
	\centering	
	\captionof{table}{The performance in NetHEPT instance ($\epsilon = 0.2$)} \label{tab:title} 	
	\begin{tabular}{ C{0.1in} C{0.5in} C{0.5in} C{0.4in} C{0.4in} C{0.4in}}\toprule[0.4pt]
		\bf k & \bf diffusion model  & \bf time\_avg(s) & \bf ISE & \bf time\_OJ & \bf OPT\_OJ\\ \midrule
		 & IC & 1.32 & 324.16 & 10.04 & 324.16 \\ 
		5 & LT & 1.04 & 393.96 & 2.557 & 392.98 \\ \midrule
	     & IC & 1.37 & 1291.89 & 53.98 & 1298.10 \\ 
		50 & LT & 1.21 & 1699.58 & 8.398 & 1702.00 \\
		\bottomrule[1.25pt]
		\end {tabular}\par
		\bigskip 
		\centering{}
		
	\end{minipage}

\begin{minipage}{\linewidth}
	\centering	
	\captionof{table}{The performance in NetHEPT instance ($\epsilon = 0.1$)} \label{tab:title} 	
	\begin{tabular}{ C{0.1in} C{0.5in} C{0.5in} C{0.4in} C{0.4in} C{0.4in}}\toprule[0.4pt]
		\bf k & \bf diffusion model  & \bf time\_avg(s) & \bf ISE & \bf time\_OJ & \bf OPT\_OJ\\ \midrule
		 & IC & 3.34 & 323.31 & 10.04 & 324.16 \\ 
		5 & LT & 2.78 & 393.72 & 2.557 & 392.98 \\ \midrule
		 & IC & 3.99 & 1295.20 & 53.98 & 1298.10 \\ 
		50 & LT & 2.91 & 1701.19 & 8.398 & 1702.00 \\
		\bottomrule[1.25pt]
		\end {tabular}\par
		\bigskip 
		\centering{}
		
\end{minipage}

\begin{minipage}{\linewidth}
	\centering	
	\captionof{table}{The performance in NetHEPT instance ($\epsilon = 0.05$)} \label{tab:title} 	
	\begin{tabular}{ C{0.1in} C{0.5in} C{0.5in} C{0.4in} C{0.4in} C{0.4in}}\toprule[0.4pt]
		\bf k & \bf diffusion model  & \bf time\_avg(s) & \bf ISE & \bf time\_OJ & \bf OPT\_OJ\\ \midrule
		 & IC & 10.04 & 323.72 & 10.04 & 324.16 \\ 
		5 & LT & 7.87 & 393.72 & 2.557 & 392.98 \\ \midrule
		 & IC & 12.34 & 1297.22 & 53.98 & 1298.10 \\ 
		50 & LT & 8.52 & 1702.00 & 8.398 & 1702.00 \\
		\bottomrule[1.25pt]
		\end {tabular}\par
		\bigskip 
		\centering{}
\end{minipage}

\begin{minipage}{\linewidth}
	\centering	
	\captionof{table}{The performance in larger instances ($\epsilon = 0.1$)} \label{tab:title} 	
	\begin{tabular}{ C{0.6in} C{0.3in} C{0.5in} C{0.4in} }\toprule[0.4pt]
		\bf Instance & \bf k & \bf diffusion model & \bf time(s) \\ \midrule
		& & IC & 6.25 \\
		& 5 & LT & 4.17 \\ 
		& & IC & 8.75 \\
		Epinion & 50 & LT & 6.72 \\ \midrule
		& & IC & 55.12 \\
		& 5 & LT & 15.41 \\
		& & IC & 70.51 \\
		Twitter & 50 & LT & 16.70\\
		\bottomrule[1.25pt]
		\end {tabular}\par
		\bigskip
		\centering{}
		
\end{minipage}

\bigbreak

\par 






\subsection{Conclusion}
% use section* for acknowledgment
According to the experiment results, it proves that my code can get the same quality solutions in less time. To conclude, the algorithm proposed in \cite{IMM} is creative and efficient. Not only reversing graph to solve \textit{influence maximization problem} is proposed, but also it calculates the lower bounds of sampling size saving computation cost. The obvious disadvantage is that it still needs massive sampling which is memory consuming. This project greatly helps me to understand the extremely important role of data structure and code construction when I completed some tricky idea in my code. If there is a chance, I would like to contribute some works in \textit{infulence maximization} problem. 








% trigger a \newpage just before the given reference
% number - used to balance the columns on the last page
% adjust value as needed - may need to be readjusted if
% the document is modified later
%\IEEEtriggeratref{8}
% The "triggered" command can be changed if desired:
%\IEEEtriggercmd{\enlargethispage{-5in}}

% references section

% can use a bibliography generated by BibTeX as a .bbl file
% BibTeX documentation can be easily obtained at:
% http://mirror.ctan.org/biblio/bibtex/contrib/doc/
% The IEEEtran BibTeX style support page is at:
% http://www.michaelshell.org/tex/ieeetran/bibtex/
\bibliographystyle{IEEEtran}
% argument is your BibTeX string definitions and bibliography database(s)
%\bibliography{IEEEabrv,../bib/paper}
%
% <OR> manually copy in the resultant .bbl file
% set second argument of \begin to the number of references
% (used to reserve space for the reference number labels box)
\begin{thebibliography}{1}
\bibitem{IMM}Y. Tang, Y. Shi, and X. Xiao, Influence Maximization in Near-Linear
Time, in Proceedings of the 2015 ACM SIGMOD International Conference on Management of Data - SIGMOD 15, 2015.
\end{thebibliography}





% that's all folks
\end{document}


